\section{Linear Programming}

This chapter covers the basics of linear programming. For anyone who has already
studied linear programming note that I do not present it using tableaus.

\subsection{Motivation}

In the context of mathematics, "programming" and optimization are interchangeable.
Linear programming is concerned with maximizing functions that are linear in its
variables.\\

Linear programming has many applications in real life, so a small example will
serve as motivation.

Say you were a farmer with $100 \text{m}^2$ of land, 10 buckets of wheat grain
and 20 buckets of rye, along with 100L of irrigation water. This begs the question;
how much wheat and rye should you plant to maximize your profits?

To answer the question, we require more information. How much water does it take
to grow wheat and rye? Assuming the answer is 10L/m$^2$ and 7L/m$^2$  respectively, we can
write the corresponding linear program as:

% TODO: Insert example mathematics
\begin{align}
\label{twoindex}
\textrm{min}\quad  \sum_{i=1}^n \sum_{j=1}^n c_{ij}x_{ij} & & \\
\textrm{s.t.}\quad \sum_{i=1, i\neq j}^n x_{ij} & =& 1,\hspace{0.5cm}\forall j\in V \label{twoindexarrive}\\
\quad \sum_{j=1, j\neq i}^n x_{ij} & =& 1, \hspace{0.5cm} \forall i\in V \label{twoindexleave}\\
\quad \sum_{(i,j) \in S}^n x_{ij} & \leq& |S| - 1, \hspace{0.5cm} \forall S\subset V, \hspace{0.25cm} 2\leq |S| \leq n \label{twoindexsec}\\
x_{ij}& \in & \{ 0, 1\} 
\end{align}

The solution to the farmer's problem is x somevector, and in the following section we
will see how the result was obtained. 

\subsection{Simplex}

In this section I present the \simplex algorithm (or method). It was orignally developed by
Dantzig in 1947 (need reference).

Finally, for my fellow computer scientists and other people interested in
functional programming, I have implemented \simplex in Haskell which you
can access by visiting \href{http://github.com/andreasbock/simplex}{\tt http://github.com/andreasbock/simplex}.

\subsection{Duality}\label{sec:duality}

\subsubsection{Duality gap}
