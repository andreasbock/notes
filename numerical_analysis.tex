\chapter{Numerical Analysis}\label{chapter:numerical_analysis}
\localtableofcontents

\section{Introduction}

In the following we let $u$ denote the exact solution and $u_h$ the numerical approximation to some equation $L u = f$. A large part of scientific computing is
concerned with the solution of such equations. Below are the three pillars of numerical analysis related to this equation:
\begin{itemize}
\item \textbf{Convergence}: If $u - u_h \rightarrow 0$ as $h\rightarrow 0$
\item \textbf{Consistency}: If $L_h u_h - f_h \rightarrow 0$ as $h\rightarrow 0$
\item \textbf{Stability}: Does perturbations to initial conditions cause a 
    proportional perturbation in the solution, independent of $h$?
\end{itemize}

We can summarise the properties above in the Fundamental Theorem of Numerical Analysis:
\begin{theorem}\label{theorem:ftnum}
A numerical scheme that is consistent and stable is convergent.
\end{theorem}

%\subsection{Measuring convergence, consistency and stability}
%Taken from \url{http://www-users.math.umn.edu/~arnold/talks/feng-kang-print.pdf}:
Let $r_h : V \rightarrow V_h$ denote the restriction operator.

\begin{itemize}
\item $\| r_h u - u_h \|$ measures the \textbf{discretisation error}
\item $\| L_h u_h - f_h \|$ measures the \textbf{consistency error}
\item $\| L_h^{-1}\|$ measures the \textbf{stability constant}
\end{itemize}

So, theorem \ref{theorem:ftnum} is a consequence of these definitions since
\[
\| r_h u - u_h \|\leq \| L_h^{-1}\|\| L_h r_h u - f_h \|
\]

\section{Linear Systems}

Unless otherwise stated we are solving $Ax=b$.

\subsection{Direct Methods}

\subsubsection{Cholesky}

\textit{Input}: $A$ semi positive definite.\\
\textit{Factorisation}: $A=LL^T$, where $L$ is a lower triangular matrix.\\

The factorisation is unique when A is PD. It is non-unique for PSD. matrices. Another version exists, $A=LDL^*$, where $D$ is a diagonal matrix and $L_{ii}=1$. This version can be more computionally convenient.\\

\textit{Complexity}: Algorithms for Cholesky factorisations are, in general, $O(n^3)$. 
\subsubsection{LU} $A=LU$, where $L$ and $U$ are, lower and upper triangular matices, respectively.

Algorithms for LU factorisations are, in general, the order of matrix-matrix products.

\subsubsection{QR}

\textit{Input}: $A$ can be any real square matrix.\\
\textit{Factorisation}: $A = QR$, where $LQ$ is an orthogonal matrix and $R$ is an upper triangular matrix.\\

Algorithms include Gram-Schmidt, Givens rotations \& Householder transformation
algorithms and.\\

\textit{Complexity}: $O(n^3)$. 

\subsubsection{Singular Value Decomposition (SVD)}: This is a kind of matrix factorisation that relies on a single observation/proof: \emph{the image of a unit circle under any matrix is a hyperellipse}. That is, for an operator $A$, we have $A = U \Sigma V^*$ where $U$, $V$ are unitary (conj. transp. is inverse) and $\Sigma$ is diagonal.
Each of the factors on the right-hand side has a very interesting interpretation! $V^*$ rotates the right-singular vectors (i.e. its rows) to align with the unit sphere. Then, $\Sigma$ (the vector) scales the coordinates and finally $U$ rotates the ellipse.\\
This is related to eigenvalues through

\subsubsection{Schur}


\subsubsection{Other Factorisations}

This section deals with the situation where we want to factorise a matrix $M$ into
some diligenty chosen constitutents.

\subsection{Indirect Methods}

\subsubsection{Jacobi Method}

This method works by decomposing $A$ into its diagonal, $D$ and the
remaining entries $R$. A Jacobi iteration is defined as:
\[
x^{k+1} = D^{-1}1 (b  - R x^k), \qquad k\geq 0
\]

The key thing to note here is that it is necessary to store the previous
vector (and not overwrite it!)

\subsubsection{Gauss-Seidel}

This method also works by decomposition of $A$, but into a lower and upper
triangular part $L$ and $U$, respectively. The Gauss-Seidel iteration is
then defined as:
\[
x^{k+1} = L^{-1}1 (b  - U x^k), \qquad k\geq 0
\]

This is a kind of iterative solver for linear systems for which convergence is guaranteed for SPD matrices.

\subsubsection{Successive Over-relaxation (SOR)}: generalisation of Gauss-Seidel but with faster convergence

\subsubsection{Krylov Subspace Methods}

% What it is
We want to approach the linear system $Ax=b$ in a more memory-efficient manner.

% Convergence and relation to eigenvalues
Conditioning of a linear system roughly states how "bad" a solution will be as a result of approximation.
The condition number of an operator measures by how much the solution $x$ will change as a resut of $b$.
This is before round-off errors or anything!\\

When $A$ is symmetric (Hermitian) then minimiser of $f(x) = \frac12 x^T Ax + x^T b$ satisfies
$Ax=b$.
 
The condition number is related to convergence of iterative methods via Chebyshev polynomials (see Krylov solver below!)

relation with eigenvalues: We have a bound that says that the error in the k\textsuperscript{th} iteration is bounded from above
by the number of distinct eigenvalues. In particular, if all eigenvalues lie in the interval $[\lambda_{min},\,\lambda_{max}]$ then we
have 
\[
\frac{\|x-x_k\|}{\|x-x_0\|} \leq 2 \Big(\frac{\sqrt{\kappa}-1}{\sqrt{\kappa}+1}\Big)^k
\]
where $\kappa = \frac{\lambda_{max}}{\lambda_{min}}$.
\underline{\emph{This is the connection with condition number!}}

\subsection{Block Methods}

\begin{itemize}
\item \textbf{Block Jacobi}: Apply the classic Jacobi preconditioning to each partition of the variables i.e. use block-diagonal instead of diagonal.
Another variant is using several diagonals (e.g. tridiagonal).\\
Each processor solves its own subsystem, and there is no communication between the processors. Since this strategy neglects the global/implicit properties of the linear system, only a limited improvement in the number of iterations can result. On the other hand, this type of method is very parallel.
\item \textbf{Multiplicative Schwarz}: In MS, the domain is divided into overlapping subdomains on which the PDE is solved. The interior boundary conditions are
simply the interior values in the neighbouring domains.
\item \textbf{Additive Schwarz}: %$M_{AS}^{-1} =\sum_{j=1}^J R^T_j A^{-1} R_j$.
As in the Block Jacobi method, each processor solves a local subsystem. However, the local system is now augmented to include bordering variables, which belong to other processors. A certain amount of communication is now necessary, and the convergence improvement can by much higher. In contrast to MS, AS lends itself better to parallel implementation. This is because we use different interior boundary conditions. In MS, we
just apply the values of the most recent updated neighbour as artificial BC. In AS, we take the previous iteration's values on neighbouring subdomains and their average
is set as the artificial boundary condition.\\

Mathematically we can write an AS iteration in residual form as follows:
\[
u^n = u^{n-1} + \sum_{i=1}^P R^T_i L^{-1}_i R_i (f - L u^{n-1})
\]
where $P$ is the number of patches, and $R$ the restriction operator.\\

\textbf{Is there \emph{any} difference between block Jacobi and AS?}
\end{itemize}

The next section describes how we can multiply the equation $Ax=b$ with another matrix to better the
conditioning of the overall system. We shall also see hwo some of the methods so far can actually
be used a preconditioner.

\section{Preconditioning}

We just saw that the convergence of Krylov subspace methods is very dependent on the 
eigenvalues of the linear operator $A$.\\

Preconditioning falls into two categories. Let $P$ be a preconditioner:
\begin{itemize}
\item \textbf{Left preconditioning}: We solve $P^{-1}(Ax - b) = 0$.
\item \textbf{Right preconditioning}: We solve $A P^{-1} P x = b$ (i.e. first solving $A P^{-1} y = b$ and then $Px = y$).
\end{itemize}

\textbf{The point is that $P^{-1}A$ has smaller conditioner number than A!}\\

We provide a list of common preconditioners
\begin{itemize}
\item \textbf{Jacobi}: We select $P = \text{diag} A$.
\item \textbf{Incomplete Cholesky or LU factorisation}:
\item \textbf{Incomplete LU factorisation}:
\end{itemize}

\subsection{Block Preconditioning}

% Block structure

\begin{itemize}
\item \textbf{Block Jacobi AKA Additive Schwarz}: Apply the classic Jacobi preconditioning to each partition of the variables i.e. use block-diagonal instead of diagonal.
Another variant is using several diagonals (e.g. tridiagonal).\\
Each processor solves its own subsystem, and there is no communication between the processors. Since this strategy neglects the global/implicit properties of the linear system, only a limited improvement in the number of iterations can result. On the other hand, this type of method is very parallel.
\item \textbf{Schwarz (additive/multiplicative)}: Previously, we saw Schwarz methods for solving a linear system of equations (possibly in parallel). In
fact, sometimes it is useful to apply these as a preconditioner while using some Krylov method to solve the actual problem. It is claimed that one iteration
of of Schwarz can serve as an optimal preconditioner. Whether it be AS or MS, this preconditioner $B$ is sought by performing one iteration with a zero
initial guess. For AS we write the preconditioned residual vector $v$ as
\[
v = Br = B(b-Ax) = \sum_{i=1}^P R^T_i L^{-1}_i R_i (b-Ax)
\]
while for MS we can write
\begin{align*}
u^{\frac{i}{P}} = u^{\frac{i-1}{P}} + R^T_i L^{-1}_i R_i (f - L u^{\frac{i-1}{P}}), \qquad i=1,\dots,P
\end{align*}
where $u^0 = 0$
\end{itemize}

See e.g. \texttt{http://www.netlib.org/utk/papers/iterative-survey/node6.html}\\

So in AS we define $A_i = R_i A R_i^T$, where $R_i$ is the restriction operator
and $A$ is the matrix in the linear system we want to solve. Inverting each $A_i$
locally and re-applying the restriction operators from local to global we get a
preconditioner agreeing with $A$ in dimension for the local system $i$:
\begin{align*}
B_i & = R_i^T A_i^{-1} R_i\\
    & = R_i^T \big(R_i A_i R_i^T \big)^{-1} R_i
\end{align*}

For a global preconditioner $B$ all we have to do is a summation:
\begin{align*}
B = \sum_i B_i
\end{align*}

\begin{remark}[Schwarz versus linear solvers]
Based on the descriptions above we can make a comparison between additive and multiplicative Schwarz and
the linear solvers Gauss-Seidel and Jacobi. Specifically, we can view additive Schwarz as a block Jacobi
method and multiplicative Schwarz as block Gauss-Seidel.\\
The similarities can be seen as follows: additive Schwarz requires all the values of the previous iteration
in order to update the boundary conditions but can be done in parallel. On the other hand, multiplicative
Schwarz cannot be run in parallel but has the advantage of not requiring storage of all solutions!\\

\textbf{\emph{NB:}} Although we compare Schwarz methods to linear solvers, they are not exactly the
same because of the choice of overlap!
\end{remark}

\section{Multigrid}

The idea behind multigrid is to generate a sequence of increasingly coarser meshes
on which you solve a system of linear equations. Multigrid is used \emph{in conjunction}
with a solver, although, as we shall see momentarity, in this context they are usually
referred to as \emph{smoothers}.\\

A property of the solvers we have seen so far is that they abate the residual starting
with the 'highest peaks first'. In other words, high frequency error (oscillations) are
removed first which is why the solvers are called smoothers. Supposing a smoother has removed
some of the error on a fine mesh so that our error now looks like a very smooth function.
If you were to take such a function and interpolate it onto a coarser mesh it would
become more irregular, or non-smooth, as a result. Hence we can re-apply a smoother to
remove even more error! Repeating this we now see what is meant above by 'generating a
sequence of coarser meshes': Using the smoothing property of iterative methods we can
remove error much faster than just applying an iterative method on a single mesh (assuming
we know how to interpolate a numerical solution between meshes!)

%\subsection{Algebraic Multigrid}
%\subsection{Geometric Multigrid}

\section{Fast Linear Algebra}

\subsection{Fast Fourier Transform}
\subsection{Fast Multipole Method}

\section{Misc}

- Get to the bottom of inf-sup conditions and how they relate to GEVPs -
Poincaré constant is the largest eigenvaue of Hx = C Dx? Since H = M + D (M is
mass) then Dx = Mx/(C-1) so  1/(C-1) is EV of D up to constant M (dep on domain)
- Estimating PC constant: consult min/max EVs of the operator ($G^-1 o E$)
(operators stemming from each norm).


\begin{itemize}
\item \textbf{Schur complement}: this is basically a way to solve a system of linear equations of size $(p+q)^2$ by instead inverting a $p^2$ and $q^2$ matrix. ii)
it can also be used in domain decomposition methods (which result in exactly a system of linear equations). This allows for parallelisation in order to find
interior Dirichlet boundary conditions and then use this to solve each subproblem.
\item \textbf{Conditioning (for $A$)}: We denote this $\kappa(A)$ which, depending on your context, is roughly the "derivative" of $A$.
\begin{itemize}
\item \textbf{Matrix-vector}: $\kappa(A) = \| A \| \frac{\|x\|}{\| Ax \|}$.
\item \textbf{Matrix (this is called the \emph{condition number})}: $\kappa(A) = \| A^{-1} \| \| A \|$. If the norm is $\ell^2$, we write $\kappa(A) = \frac{\sigma_1}{\sigma_m}$
\end{itemize}
This is important because we essentially stand to lose $\log_{10} \kappa(A)$ digits when approximating a solution to $Ax=b$.
The condition number is related to convergence of iterative methods via Chebyshev polynomials (see Krylov solver below!)

\item \textbf{Stiffness}: Roughly means how "spread out" the eigenvalues are
\end{itemize}
